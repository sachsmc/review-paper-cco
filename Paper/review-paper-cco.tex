\documentclass[11pt]{article}
\usepackage{hyperref}
\usepackage[margin=35mm]{geometry}



\title{\bigskip \bigskip Statistical Principles for Omics-based Clinical Trials}

%\author{true\\true}

\author{\Large Michael C Sachs\vspace{0.05in} \\ \normalsize\emph{National Cancer Institute} \\ \footnotesize \url{maito:michael.sachs@nih.gov}\vspace*{0.2in}\\  \and \Large Lisa M McShane\vspace{0.05in} \\ \normalsize\emph{National Cancer Institute} \\ \footnotesize \url{maito:McShaneL@CTEP.NCI.NIH.gov}\vspace*{0.2in}\\ }

%\author{Michael C Sachs (National Cancer Institute) \and Lisa M McShane (National Cancer Institute)}

\date{\footnotesize September 2014. Incomplete Draft. Please do not cite without permission.}
\linespread{2}

\begin{document}  
		




\maketitle


\begin{abstract}

\noindent High-throughput techologies enable the measurement of a large number of
molecular characteristics from a small tissue sample. High-dimensional
molecular information (referred to as omics data) offers the possibility
of predicting the future outcome of a patient (prognosis) and predicting
the likely response to a specific treatment (prediction). Embedded in
the vast amount of data is the hope that there exists some signal that
will enable practicioners to deliver therapy personalized to the
molecular profile of a tumor, thereby improving health outcomes. The
challenges are to determine that the omics assays are valid and
reproducible in a clinical setting, to develop a valid and optimal
omics-based test that algorithmically determines the optimal treatment
regime, to evaluate that test in a powerful and unbiased manner, and
finally to demonstrate clinical utility: that the test under study
improves clinical outcome as compared to not using the test. We review
the statistical considerations involved in each of these stages,
specifically dealing with the challenges of high-dimensional, omics
data.

\smallskip
\noindent \textbf{Keywords.} genomics; personalized medicine; predictive biomarker; statistics

\end{abstract}


\section{Introduction}\label{introduction}

Omics technolgies that generate a large amount of molecular data about a
cancerous tumor have the potential to provide accurate predictions of a
patient's prognosis and their response to a specific treatment regime.
The idea of omics-based biomarkers is that distinct tumor types can be
identified using the multi-dimensional molecular data leading to
treatment decisions personalized to that tumor type. An omics-based test
can guide the decisions to treat or not to treat and help identify the
particular therapy most likely to work. The challenge is to identify and
demonstrate definitively that the use of an omics-based decision rule
improves clinical outcomes in a patient population.

The prognosis of a patient is their expected clinical outcome. An
omics-based test can be used to predict a patient's prognosis. A test
that provides accurate predictions of prognosis, regardless of
treatment, is referred to as a prognostic biomarker. A predictive
omics-test is one that accurately predicts disease outcomes with the
application of specific interventions. Predictive markers are therefore
useful for the selection among two or more treatment options.
Statistically, a prognostic test is strongly associated with clinical
outcome and a predictive test modifies the association between treatment
and clinical outcome (interaction). The two are not mutually exclusive,
however. It is uncommon for a test to be purely predictive (1), and
prognostic tests can be used to inform treatment decisions.

A patient's prognosis can be used to determine the type and intensity of
medical treatment, or whether to treat at all. Endopredict (2) is an
omics-based test test that is used to determine the likelihood of
distant recurrence in ER-positive, HER2-negative breast cancer. The test
has been shown to accurately predict prognosis, and is therefore useful
for guiding treatment decisions, determining eligibility for trials, and
making disease-management decisions. Endopredict as a prognostic test
has been rigorously evaluated and shown to be clinically valid, even
thought it does not predict response to any specific therapy.

The goal of this paper is to review the path to definitively evaluating
an omics-based test for prognosis or prediction of treatment response.
We assume that the patient population is well-defined, and that there
may exist targeted therapies for a subset of that population. High
dimensional omics data can be used to identify specific molecular
targets as potential mechnisms for drug development, however the use of
omics technologies for drug development is beyond the scope of this
review.

The long road to implementing a test in a clinical trial starts with
analytical validation, that is, demonstrating that the omics-based assay
accurately and reproducibly measures the molecular quantities. After the
assay performance is established comes the test development and
preliminary evaluation. This involves reducing the high-dimensional data
into a one-dimensional quantity that will be used to make a decision.
This one-dimensional quantity is often a risk score: an estimate of the
probability of a speicific clinical outcome. It is necessary to
establish the clinical validity of this risk score, that is, demonstrate
that the risk score is independently associated with clinical outcome.
Care must be taken to completely separate the development of the risk
score from the evaluation, otherwise estimates can be optimistically
biased. Finally, the risk score must be translated into a binary
decision, often using a threshold. It remains to demonstrate that the
use of the test to make this decision improves patient outcomes.

\section{Analytical validation}\label{analytical-validation}

Analytical validation of an assay involves evaluating the performance of
the measurement in terms of accuracy, bias, and precision under a
variety of conditions. Conditions are things like preanalytic factors
such as specimen quality, specimen collection, storage, and processing
procedures, and technical aspects such as laboratory technician and
batch effects from reagent lots or other assay materials. The
high-dimensional nature of omics data makes it very difficult to assess
each of the hundreds or thousands of outputs from a single assay. In
developing a omics-based signature that only uses a subset of the
components of a high-dimensional assay, one can analytically validate
the final signature alone. However, prior to developing the signature,
one must develop detailed standard operating procedures for specimen
handling and processing to ensure a baseline level of validity.

Do: develop criteria for the rejection of poor-quality specimens.
Percent tumor, necrosis, etc.

Do: filter features based on QC prior to development

Do: assess the impact of sample and specimen handling. (3--7)

Do: assess the impact of lots and batch effects. Bias, accuracy. (8,9)

Do: minimize the impact of technical aspects to greatest extent possible
by developing detailed SOP.

\section{Test development and preliminary
evaluation}\label{test-development-and-preliminary-evaluation}

Study design: consider retrospective (10)

Don't: confound technical factors with clinical outcomes. (11,12)

Do: maintain strict separation between development and evaluation.

Do: cross validation if you have a data-sparse setting. (1,13--15)

Don't: use convoluted methods leading to overfitting.

Don't: do partial resubstitution

Compare: feature filtering based on association with outcome,
regularization. (16,17)

Do: consider all available methods, model averaging. Hard to determine
best method in advance.

Don't: rely on clustering to yield good predictions of outcome.

\section{Demonstrating clinical
utility}\label{demonstrating-clinical-utility}

Do: define the clinical use (18)

Do: use a valid and interpretable statistical method appropriate to that
use (19--21)

Do: Design your study appropriately to answer the clinical question
definitively (22--33)

Do: Power your trial appropriately (34,35)

Don't: make these mistakes (36--38)

\section{Concluding remarks}\label{concluding-remarks}

Do: follow reporting criteria (39--41)

\section{References}\label{references}

\setlength{\parindent}{0pt}

1. McShane LM, Polley M-YC. Development of omics-based clinical tests
for prognosis and therapy selection: The challenge of achieving
statistical robustness and clinical utility. Clinical Trials. SAGE
Publications; 2013;10(5):653--65.

2. Filipits M, Rudas M, Jakesz R, Dubsky P, Fitzal F, Singer CF, et al.
A new molecular predictor of distant recurrence in eR-positive,
hER2-negative breast cancer adds independent information to conventional
clinical risk factors. Clinical Cancer Research. American Association
for Cancer Research; 2011;17(18):6012--20.

3. Werner M, Chott A, Fabiano A, Battifora H. Effect of formalin tissue
fixation and processing on immunohistochemistry. The American journal of
surgical pathology. LWW; 2000;24(7):1016--9.

4. Srinivasan M, Sedmak D, Jewell S. Effect of fixatives and tissue
processing on the content and integrity of nucleic acids. The American
journal of pathology. Elsevier; 2002;161(6):1961--71.

5. Maldegem F van, Wit M de, Morsink F, Musler A, Weegenaar J, Noesel CJ
van. Effects of processing delay, formalin fixation, and
immunohistochemistry on rNA recovery from formalin-fixed
paraffin-embedded tissue sections. Diagnostic Molecular Pathology. LWW;
2008;17(1):51--8.

6. Specht K, Richter T, M{ü}ller U, Walch A, Werner M, H{ö}fler H.
Quantitative gene expression analysis in microdissected archival
formalin-fixed and paraffin-embedded tumor tissue. The American journal
of pathology. Elsevier; 2001;158(2):419--29.

7. Iwamoto KS, Mizuno T, Ito T, Akiyama M, Takeichi N, Mabuchi K, et al.
Feasibility of using decades-old archival tissues in molecular
oncology/epidemiology. The American journal of pathology. American
Society for Investigative Pathology; 1996;149(2):399.

8. Pennello GA. Analytical and clinical evaluation of biomarkers assays:
When are biomarkers ready for prime time? Clinical Trials. SAGE
Publications; 2013;1740774513497541.

9. Isler JA, Vesterqvist OE, Burczynski ME. Analytical validation of
genotyping assays in the biomarker laboratory. Future Medicine Ltd;
2007;

10. Simon RM, Paik S, Hayes DF. Use of archived specimens in evaluation
of prognostic and predictive biomarkers. Journal of the National Cancer
Institute. Oxford University Press; 2009;101(21):1446--52.

11. Leek JT, Scharpf RB, Bravo HC, Simcha D, Langmead B, Johnson WE, et
al. Tackling the widespread and critical impact of batch effects in
high-throughput data. Nature Reviews Genetics. Nature Publishing Group;
2010;11(10):733--9.

12. Soneson C, Gerster S, Delorenzi M. Batch effect confounding leads to
strong bias in performance estimates obtained by cross-validation. PloS
one. Public Library of Science; 2014;9(6):e100335.

13. Moons KG, Kengne AP, Woodward M, Royston P, Vergouwe Y, Altman DG,
et al. Risk prediction models: I. development, internal validation, and
assessing the incremental value of a new (bio) marker. Heart. BMJ
Publishing Group Ltd; British Cardiovascular Society;
2012;98(9):683--90.

14. Moons KG, Kengne AP, Grobbee DE, Royston P, Vergouwe Y, Altman DG,
et al. Risk prediction models: II. external validation, model updating,
and impact assessment. Heart. BMJ Publishing Group Ltd; British
Cardiovascular Society; 2012;heartjnl--l2011.

15. Polley M-YC, Freidlin B, Korn EL, Conley BA, Abrams JS, McShane LM.
Statistical and practical considerations for clinical evaluation of
predictive biomarkers. Journal of the National Cancer Institute. Oxford
University Press; 2013;105(22):1677--83.

16. Bair E, Tibshirani R. Semi-supervised methods to predict patient
survival from gene expression data. PLoS biology. Public Library of
Science; 2004;2(4):e108.

17. Hastie T, Tibshirani R, Friedman J, Hastie T, Friedman J, Tibshirani
R. The elements of statistical learning. Springer; 2009.

18. Mandrekar SJ, Sargent DJ. Predictive biomarker validation in
practice: Lessons from real trials. Clinical trials. SAGE Publications;
2010;7(5):567--73.

19. Janes H, Brown MD, Huang Y, Pepe MS. An approach to evaluating and
comparing biomarkers for patient treatment selection. The international
journal of biostatistics. 2014;10(1):99--121.

20. Pepe MS. Problems with risk reclassification methods for evaluating
prediction models. American journal of epidemiology. Oxford University
Press; 2011;kwr013.

21. Hilden J, Gerds TA. A note on the evaluation of novel biomarkers: Do
not rely on integrated discrimination improvement and net
reclassification index. Statistics in medicine. John Wiley \& Sons, Ltd;
2013;

22. Freidlin B, Korn EL. Biomarker enrichment strategies: Matching trial
design to biomarker credentials. Nature Reviews Clinical Oncology.
Nature Publishing Group; 2014;11(2):81--90.

23. Baker SG, Sargent DJ. Designing a randomized clinical trial to
evaluate personalized medicine: A new approach based on risk prediction.
Journal of the National Cancer Institute. Oxford University Press; 2010;

24. Baker SG, Kramer BS, Sargent DJ, Bonetti M. Biomarkers, subgroup
evaluation, and clinical trial design. Discovery medicine.
2012;13(70):187--92.

25. Brannath W, Zuber E, Branson M, Bretz F, Gallo P, Posch M, et al.
Confirmatory adaptive designs with bayesian decision tools for a
targeted therapy in oncology. Statistics in medicine. Wiley Online
Library; 2009;28(10):1445--63.

26. Denne JS, Pennello G, Zhao L, Chang S-C, Althouse S. Identifying a
subpopulation for a tailored therapy: Bridging clinical efficacy from a
laboratory-developed assay to a validated in vitro diagnostic test kit.
Statistics in Biopharmaceutical Research. Taylor \& Francis;
2014;6(1):78--88.

27. Eng KH. Randomized reverse marker strategy design for prospective
biomarker validation. Statistics in medicine. Wiley Online Library;
2014;

28. Freidlin B, Jiang W, Simon R. The cross-validated adaptive signature
design. Clinical Cancer Research. AACR; 2010;16(2):691--8.

29. Freidlin B, McShane LM, Polley M-YC, Korn EL. Randomized phase iI
trial designs with biomarkers. Journal of Clinical Oncology. American
Society of Clinical Oncology; 2012;30(26):3304--9.

30. Freidlin B, Korn EL, Gray R. Marker sequential test (maST) design.
Clinical Trials. SAGE Publications; 2013;1740774513503739.

31. Jiang W, Freidlin B, Simon R. Biomarker-adaptive threshold design: A
procedure for evaluating treatment with possible biomarker-defined
subset effect. Journal of the National Cancer Institute. Oxford
University Press; 2007;99(13):1036--43.

32. Mandrekar SJ, Sargent DJ. Clinical trial designs for predictive
biomarker validation: Theoretical considerations and practical
challenges. Journal of Clinical Oncology. American Society of Clinical
Oncology; 2009;27(24):4027--34.

33. Morita S, Yamamoto H, Sugitani Y. Biomarker-based bayesian
randomized phase iI clinical trial design to identify a sensitive
patient subpopulation. Statistics in medicine. John Wiley \& Sons, Ltd;
2014;

34. Mackey HM, Bengtsson T. Sample size and threshold estimation for
clinical trials with predictive biomarkers. Contemporary clinical
trials. Elsevier; 2013;36(2):664--72.

35. Peterson B, George SL. Sample size requirements and length of study
for testing interaction in a 1\(\times\)\textless{} i\textgreater{}
k\textless{}/i\textgreater{} factorial design when time-to-failure is
the outcome. Controlled clinical trials. Elsevier; 1993;14(6):511--22.

36. Lee S. Mistakes in validating the accuracy of a prediction
classifier in high-dimensional butsmall-sample microarray data.
Statistical methods in medical research. SAGE Publications; 2008;

37. Sargent DJ, Mandrekar SJ. Statistical issues in the validation of
prognostic, predictive, and surrogate biomarkers. Clinical Trials. SAGE
Publications; 2013;10(5):647--52.

38. Simon R, Radmacher MD, Dobbin K, McShane LM. Pitfalls in the use of
dNA microarray data for diagnostic and prognostic classification.
Journal of the National Cancer Institute. Oxford University Press;
2003;95(1):14--8.

39. Bouwmeester W, Zuithoff NP, Mallett S, Geerlings MI, Vergouwe Y,
Steyerberg EW, et al. Reporting and methods in clinical prediction
research: A systematic review. PLoS medicine. Public Library of Science;
2012;9(5):e1001221.

40. Janssens AC, Ioannidis J, Bedrosian S, Boffetta P, Dolan SM, Dowling
N, et al. Strengthening the reporting of genetic risk prediction studies
(gRIPS): Explanation and elaboration. European journal of clinical
investigation. Wiley Online Library; 2011;41(9):1010--35.

41. McShane LM, Cavenagh MM, Lively TG, Eberhard DA, Bigbee WL, Williams
PM, et al. Criteria for the use of omics-based predictors in clinical
trials: Explanation and elaboration. BMC medicine. BioMed Central Ltd;
2013;11(1):220.

\end{document}