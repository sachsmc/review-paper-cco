\documentclass[11pt]{article}
\usepackage{hyperref}
\usepackage[margin=35mm]{geometry}



\title{\bigskip \bigskip Statistical Principles for Omics-based Clinical Trials}

%\author{true}

\author{\Large Michael C Sachs\vspace{0.05in} \\ \normalsize\emph{National Cancer Institute} \\ \footnotesize \url{maito:michael.sachs@nih.gov}\vspace*{0.2in}\\ }

%\author{Michael C Sachs (National Cancer Institute)}

\date{\footnotesize October 2014. Incomplete Draft. Please do not cite without permission.}
\linespread{2}

\begin{document}  
		




\maketitle


\begin{abstract}

\noindent High-throughput techologies enable the measurement of a large number of
molecular characteristics from a small tissue sample. High-dimensional
molecular information (referred to as omics data) offers the possibility
of predicting the future outcome of a patient (prognosis) and predicting
the likely response to a specific treatment (prediction). Embedded in
the vast amount of data is the hope that there exists some signal that
will enable practicioners to deliver therapy personalized to the
molecular profile of a tumor, thereby improving health outcomes. The
challenges are to determine that the omics assays are valid and
reproducible in a clinical setting, to develop a valid and optimal
omics-based test that algorithmically determines the optimal treatment
regime, to evaluate that test in a powerful and unbiased manner, and
finally to demonstrate clinical utility: that the test under study
improves clinical outcome as compared to not using the test. We review
the statistical considerations involved in each of these stages,
specifically dealing with the challenges of high-dimensional, omics
data.

\smallskip
\noindent \textbf{Keywords.} genomics; personalized medicine; predictive biomarker; statistics

\end{abstract}


\section{Introduction}\label{introduction}

Omics technolgies that generate a large amount of molecular data about a
cancerous tumor have the potential to provide accurate predictions of a
patient's prognosis and predictions of their response to a specific
treatment regime. The idea of omics-based biomarkers is that distinct
tumor types can be identified using the multi-dimensional molecular data
leading to treatment decisions personalized to that tumor type. An
omics-based test can guide the decisions to treat or not to treat and
help identify the particular therapy most likely to work. The challenge
is to identify and demonstrate definitively that the use of an
omics-based test improves clinical outcomes in a patient population.

An omics-based test can be used to predict a patient's prognosis, which
is their expected clinical outcome. A test that provides accurate
predictions of prognosis, regardless of treatment, is referred to as a
prognostic biomarker. A predictive omics test is one that accurately
predicts disease outcomes with the application of specific
interventions. Predictive markers are therefore useful for the selection
among two or more treatment options. Statistically, a prognostic test is
strongly associated with clinical outcome and a predictive test modifies
the association between treatment and clinical outcome (interaction).
High dimensional omics data can be used to identify specific molecular
targets as potential mechnisms for drug development, however the use of
omics technologies for drug development is beyond the scope of this
review.

The path from development to definitively evaluating an omics-based test
for prognosis or prediction of treatment response is long and arduous.
Often, the end goal is to develop a test suitable for use in a clinical
trial for guiding treatment. The oncology literature is full of reports
that develop and/or evaluate omics-based tools for prognosis and
prediction. Developing a simple test based on high-dimensional omics
data can be complex and often uses novel statistical methods. Definitive
evaluation of a prognostic or predictive test is costly and rife with
methodological pitfalls. We aim to review such issues, giving you the
resources to ask the right questions when critically weighing the
evidence presented in a report of an omics-based study. Ultimately, as a
practicing oncologist the question is: ``Is this omics-based test
something I want to use to improve patient care?''.

The long road to implementing a test in a practice starts with
analytical validation, that is, demonstrating that the omics-based assay
accurately and reproducibly measures the molecular quantities. After the
assay performance is established comes the test development and
preliminary evaluation. This involves reducing the high-dimensional data
into a one-dimensional quantity that will be used to make a decision.
This one-dimensional quantity is often a risk score: an estimate of the
probability of a speicific clinical outcome. It is necessary to
establish the clinical validity of this risk score, that is, demonstrate
that the risk score is independently associated with clinical outcome.
Care must be taken to completely separate the development of the risk
score from the evaluation, otherwise estimates can be optimistically
biased. Finally, the risk score must be translated into a binary
decision, often using a threshold. It remains to demonstrate that the
use of the test to make this decision improves patient outcomes.

The following sections specify questions you should ask while reading a
report of an omics-based clinical study. We review the importance of
such questions, and common pitfalls to watch for. If you are reporting
on an omics-based trial, answers to these questions should be made clear
to the reader. Formal efforts to guide reporting have been developed,
such as the REMARK checklist (1), the GRIPS statement (2), and a third
guideline article that lacks an acronym (3). Our review reflects these
efforts through the readers' lens.

\section{What is the intended clinical
use?}\label{what-is-the-intended-clinical-use}

Do: define the clinical use (4)

As with all clinical studies, the end goal is to improve patient care.
Omics studies are no different, and a clear statement of the intended
clinical use of the omics-test should be prominent.

\section{What is the patient population of
interest?}\label{what-is-the-patient-population-of-interest}

\section{Is the omics assay valid?}\label{is-the-omics-assay-valid}

Analytical validation of an assay involves evaluating the performance of
the measurement in terms of accuracy, bias, and precision under a
variety of conditions. Conditions are things like preanalytic factors
such as specimen quality, specimen collection, storage, and processing
procedures, and technical aspects such as laboratory technician and
batch effects from reagent lots or other assay materials. The
high-dimensional nature of omics data makes it very difficult to assess
each of the hundreds or thousands of outputs from a single assay. In
developing a omics-based signature that only uses a subset of the
components of a high-dimensional assay, one can analytically validate
the final signature alone. However, prior to developing the signature,
one must develop detailed standard operating procedures for specimen
handling and processing to ensure a baseline level of validity.

Do: develop criteria for the rejection of poor-quality specimens.
Percent tumor, necrosis, etc.

Do: filter features based on QC prior to development

Do: assess the impact of sample and specimen handling. (5--9)

Do: assess the impact of lots and batch effects. Bias, accuracy. (10,11)

Do: minimize the impact of technical aspects to greatest extent possible
by developing detailed SOP.

\section{What does the omics-test
do?}\label{what-does-the-omics-test-do}

Does the test provide a continuous score or a binary classification?

How are the features of the omics assay translated into a clinically
meaningful quantity?

Compare: feature filtering based on association with outcome,
regularization. (12,13)

Do: consider all available methods, model averaging. Hard to determine
best method in advance.

Don't: rely on clustering to yield good predictions of outcome.

\section{On what samples was the test
developed?}\label{on-what-samples-was-the-test-developed}

Study design: consider retrospective (14)

Don't: confound technical factors with clinical outcomes. (15,16)

Do: maintain strict separation between development and evaluation.

Do: cross validation if you have a data-sparse setting. (17--20)

Don't: use convoluted methods leading to overfitting.

\section{On what samples is the test being
evaluated?}\label{on-what-samples-is-the-test-being-evaluated}

Do: define the clinical use (4)

Do: Design your study appropriately to answer the clinical question
definitively (21--32)

Do: Power your trial appropriately (33,34)

Don't: do partial resubstitution

\section{Are valid methods being used to evalute the
test?}\label{are-valid-methods-being-used-to-evalute-the-test}

Bad: IDI or net reclassification (35,36)

Bad: Comparing AUCs for regression models (37)

Good: comprehensive and pre-specified approach (38)

\section{Are the development and evaluation samples strictly
separated?}\label{are-the-development-and-evaluation-samples-strictly-separated}

This issue has come up in previous sections, yet this error occurs so
frequently that it needs to be highlighted in its own section. The
evaluation sample for the assessment of a prognostic or predictive test
needs to be completely independent from the development sample. This is
especially true for omics-based tests, whose developement is often
complex and convoluted. Any imformation from the evaluation sample that
leaks into the development sample can bias the results, making tests
appear better than they truly are.

Don't: do partial resubstitution

Don't: make these mistakes (39--41)

\section{Concluding remarks}\label{concluding-remarks}

Do: follow reporting criteria (1--3,42)

\section{References}\label{references}

\setlength{\parindent}{0pt}

1. Altman DG, McShane LM, Sauerbrei W, Taube SE. Reporting
recommendations for tumor marker prognostic studies (rEMARK):
Explanation and elaboration. BMC medicine. BioMed Central Ltd;
2012;10(1):51.

2. Janssens AC, Ioannidis J, Bedrosian S, Boffetta P, Dolan SM, Dowling
N, et al. Strengthening the reporting of genetic risk prediction studies
(gRIPS): Explanation and elaboration. European journal of clinical
investigation. Wiley Online Library; 2011;41(9):1010--35.

3. McShane LM, Cavenagh MM, Lively TG, Eberhard DA, Bigbee WL, Williams
PM, et al. Criteria for the use of omics-based predictors in clinical
trials: Explanation and elaboration. BMC medicine. BioMed Central Ltd;
2013;11(1):220.

4. Mandrekar SJ, Sargent DJ. Predictive biomarker validation in
practice: Lessons from real trials. Clinical trials. SAGE Publications;
2010;7(5):567--73.

5. Werner M, Chott A, Fabiano A, Battifora H. Effect of formalin tissue
fixation and processing on immunohistochemistry. The American journal of
surgical pathology. LWW; 2000;24(7):1016--9.

6. Srinivasan M, Sedmak D, Jewell S. Effect of fixatives and tissue
processing on the content and integrity of nucleic acids. The American
journal of pathology. Elsevier; 2002;161(6):1961--71.

7. Maldegem F van, Wit M de, Morsink F, Musler A, Weegenaar J, Noesel CJ
van. Effects of processing delay, formalin fixation, and
immunohistochemistry on rNA recovery from formalin-fixed
paraffin-embedded tissue sections. Diagnostic Molecular Pathology. LWW;
2008;17(1):51--8.

8. Specht K, Richter T, M{ü}ller U, Walch A, Werner M, H{ö}fler H.
Quantitative gene expression analysis in microdissected archival
formalin-fixed and paraffin-embedded tumor tissue. The American journal
of pathology. Elsevier; 2001;158(2):419--29.

9. Iwamoto KS, Mizuno T, Ito T, Akiyama M, Takeichi N, Mabuchi K, et al.
Feasibility of using decades-old archival tissues in molecular
oncology/epidemiology. The American journal of pathology. American
Society for Investigative Pathology; 1996;149(2):399.

10. Pennello GA. Analytical and clinical evaluation of biomarkers
assays: When are biomarkers ready for prime time? Clinical Trials. SAGE
Publications; 2013;1740774513497541.

11. Isler JA, Vesterqvist OE, Burczynski ME. Analytical validation of
genotyping assays in the biomarker laboratory. Future Medicine Ltd;
2007;

12. Bair E, Tibshirani R. Semi-supervised methods to predict patient
survival from gene expression data. PLoS biology. Public Library of
Science; 2004;2(4):e108.

13. Hastie T, Tibshirani R, Friedman J, Hastie T, Friedman J, Tibshirani
R. The elements of statistical learning. Springer; 2009.

14. Simon RM, Paik S, Hayes DF. Use of archived specimens in evaluation
of prognostic and predictive biomarkers. Journal of the National Cancer
Institute. Oxford University Press; 2009;101(21):1446--52.

15. Leek JT, Scharpf RB, Bravo HC, Simcha D, Langmead B, Johnson WE, et
al. Tackling the widespread and critical impact of batch effects in
high-throughput data. Nature Reviews Genetics. Nature Publishing Group;
2010;11(10):733--9.

16. Soneson C, Gerster S, Delorenzi M. Batch effect confounding leads to
strong bias in performance estimates obtained by cross-validation. PloS
one. Public Library of Science; 2014;9(6):e100335.

17. McShane LM, Polley M-YC. Development of omics-based clinical tests
for prognosis and therapy selection: The challenge of achieving
statistical robustness and clinical utility. Clinical Trials. SAGE
Publications; 2013;10(5):653--65.

18. Moons KG, Kengne AP, Woodward M, Royston P, Vergouwe Y, Altman DG,
et al. Risk prediction models: I. development, internal validation, and
assessing the incremental value of a new (bio) marker. Heart. BMJ
Publishing Group Ltd; British Cardiovascular Society;
2012;98(9):683--90.

19. Moons KG, Kengne AP, Grobbee DE, Royston P, Vergouwe Y, Altman DG,
et al. Risk prediction models: II. external validation, model updating,
and impact assessment. Heart. BMJ Publishing Group Ltd; British
Cardiovascular Society; 2012;heartjnl--l2011.

20. Polley M-YC, Freidlin B, Korn EL, Conley BA, Abrams JS, McShane LM.
Statistical and practical considerations for clinical evaluation of
predictive biomarkers. Journal of the National Cancer Institute. Oxford
University Press; 2013;105(22):1677--83.

21. Freidlin B, Korn EL. Biomarker enrichment strategies: Matching trial
design to biomarker credentials. Nature Reviews Clinical Oncology.
Nature Publishing Group; 2014;11(2):81--90.

22. Baker SG, Sargent DJ. Designing a randomized clinical trial to
evaluate personalized medicine: A new approach based on risk prediction.
Journal of the National Cancer Institute. Oxford University Press; 2010;

23. Baker SG, Kramer BS, Sargent DJ, Bonetti M. Biomarkers, subgroup
evaluation, and clinical trial design. Discovery medicine.
2012;13(70):187--92.

24. Brannath W, Zuber E, Branson M, Bretz F, Gallo P, Posch M, et al.
Confirmatory adaptive designs with bayesian decision tools for a
targeted therapy in oncology. Statistics in medicine. Wiley Online
Library; 2009;28(10):1445--63.

25. Denne JS, Pennello G, Zhao L, Chang S-C, Althouse S. Identifying a
subpopulation for a tailored therapy: Bridging clinical efficacy from a
laboratory-developed assay to a validated in vitro diagnostic test kit.
Statistics in Biopharmaceutical Research. Taylor \& Francis;
2014;6(1):78--88.

26. Eng KH. Randomized reverse marker strategy design for prospective
biomarker validation. Statistics in medicine. Wiley Online Library;
2014;

27. Freidlin B, Jiang W, Simon R. The cross-validated adaptive signature
design. Clinical Cancer Research. AACR; 2010;16(2):691--8.

28. Freidlin B, McShane LM, Polley M-YC, Korn EL. Randomized phase iI
trial designs with biomarkers. Journal of Clinical Oncology. American
Society of Clinical Oncology; 2012;30(26):3304--9.

29. Freidlin B, Korn EL, Gray R. Marker sequential test (maST) design.
Clinical Trials. SAGE Publications; 2013;1740774513503739.

30. Jiang W, Freidlin B, Simon R. Biomarker-adaptive threshold design: A
procedure for evaluating treatment with possible biomarker-defined
subset effect. Journal of the National Cancer Institute. Oxford
University Press; 2007;99(13):1036--43.

31. Mandrekar SJ, Sargent DJ. Clinical trial designs for predictive
biomarker validation: Theoretical considerations and practical
challenges. Journal of Clinical Oncology. American Society of Clinical
Oncology; 2009;27(24):4027--34.

32. Morita S, Yamamoto H, Sugitani Y. Biomarker-based bayesian
randomized phase iI clinical trial design to identify a sensitive
patient subpopulation. Statistics in medicine. John Wiley \& Sons, Ltd;
2014;

33. Mackey HM, Bengtsson T. Sample size and threshold estimation for
clinical trials with predictive biomarkers. Contemporary clinical
trials. Elsevier; 2013;36(2):664--72.

34. Peterson B, George SL. Sample size requirements and length of study
for testing interaction in a 1\(\times\)\textless{} i\textgreater{}
k\textless{}/i\textgreater{} factorial design when time-to-failure is
the outcome. Controlled clinical trials. Elsevier; 1993;14(6):511--22.

35. Hilden J, Gerds TA. A note on the evaluation of novel biomarkers: Do
not rely on integrated discrimination improvement and net
reclassification index. Statistics in medicine. John Wiley \& Sons, Ltd;
2013;

36. Pepe MS. Problems with risk reclassification methods for evaluating
prediction models. American journal of epidemiology. Oxford University
Press; 2011;kwr013.

37. Seshan VE, G{ö}nen M, Begg CB. Comparing rOC curves derived from
regression models. Statistics in medicine. Wiley Online Library;
2013;32(9):1483--93.

38. Janes H, Brown MD, Huang Y, Pepe MS. An approach to evaluating and
comparing biomarkers for patient treatment selection. The international
journal of biostatistics. 2014;10(1):99--121.

39. Lee S. Mistakes in validating the accuracy of a prediction
classifier in high-dimensional butsmall-sample microarray data.
Statistical methods in medical research. SAGE Publications; 2008;

40. Sargent DJ, Mandrekar SJ. Statistical issues in the validation of
prognostic, predictive, and surrogate biomarkers. Clinical Trials. SAGE
Publications; 2013;10(5):647--52.

41. Simon R, Radmacher MD, Dobbin K, McShane LM. Pitfalls in the use of
dNA microarray data for diagnostic and prognostic classification.
Journal of the National Cancer Institute. Oxford University Press;
2003;95(1):14--8.

42. Bouwmeester W, Zuithoff NP, Mallett S, Geerlings MI, Vergouwe Y,
Steyerberg EW, et al. Reporting and methods in clinical prediction
research: A systematic review. PLoS medicine. Public Library of Science;
2012;9(5):e1001221.

\end{document}