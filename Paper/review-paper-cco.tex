\documentclass[11pt]{article}
\usepackage{hyperref}
\usepackage[margin=35mm]{geometry}



\title{\bigskip \bigskip Statistical Principles for Omics-based Clinical Trials}

%\author{true\\true}

\author{\Large Michael C Sachs\vspace{0.05in} \\ \normalsize\emph{National Cancer Institute} \\ \footnotesize \url{michael.sachs@nih.gov}\vspace*{0.2in}\\  \and \Large Lisa M McShane\vspace{0.05in} \\ \normalsize\emph{National Cancer Institute} \\ \footnotesize \url{McShaneL@CTEP.NCI.NIH.gov}\vspace*{0.2in}\\ }

%\author{Michael C Sachs (National Cancer Institute) \and Lisa M McShane (National Cancer Institute)}

\date{\footnotesize September 2014. Incomplete Draft. Please do not cite without permission.}
\linespread{2}

\begin{document}  
		




\maketitle


\begin{abstract}

\noindent High-throughput techologies enable the measurement of a large number of
molecular characteristics from a small tissue sample. High-dimensional
molecular information (referred to as omics data) offers the possibility
of predicting the future outcome of a patient (prognosis) and predicting
the likely response to a specific treatment (prediction). Embedded in
the vast amount of data is the hope that there exists some signal that
will enable practicioners to deliver therapy personalized to the
molecular profile of a tumor, thereby improving health outcomes. The
challenges are to determine that the omics assays are valid and
reproducible in a clinical setting, to develop a valid and optimal
omics-based test that algorithmically determines the optimal treatment
regime, to evaluate that test in a powerful and unbiased manner, and
finally to demonstrate clinical utility: that the test under study
improves clinical outcome as compared to not using the test. We review
the statistical considerations involved in each of these stages,
specifically dealing with the challenges of high-dimensional, omics
data.

\smallskip
\noindent \textbf{Keywords.} genomics; personalized medidcine; predictive biomarker; statistics

\end{abstract}


\section{Introduction}\label{introduction}

Omics technolgies that generate a large amount of molecular data about a
cancerous tumor have the potential to provide accurate predictions of a
patient's prognosis and their response to a specific treatment regime.
It has long been recognized in oncology that tumors of a primary site
represent a heterogeneous collection of diseases that may have different
characteristics and respond to different therapies {[}Citation
needed{]}. The idea of omics-based biomarkers is that different tumor
types can be identified using the multi-dimensional molecular data,
leading to the development of new therapies targeting specific pathways,
or leading to treatment decisions personalized to that tumor type.

The prognosis of a patient is their expected clinical outcome. An
omics-based test can be used to predict a patient's prognosis. A test
that provides accurate predictions of prognosis, regardless of
treatment, is referred to as a prognostic biomarker. A predictive
omics-test is one that accurately predicts disease outcomes with the
application of specific interventions. Predictive markers are therefore
useful for the selection among two or more treatment options.
Statistically, a prognostic test is strongly associated with clinical
outcome and a predictive test modifies the association between treatment
and clinical outcome (interaction). The two are not mutually exclusive.
It is uncommon for a test to be purely predictive (1), and prognostic
tests can be used to inform treatment decisions.

A patient's prognosis is used to determine the type and intensity of
medical treatment. Endopredict (2) is an omics-based test test that is
used to determine the likelihood of distant recurrence in ER-positive,
HER2-negative breast cancer. The test has been shown to accurately
predict prognosis, and is therefore useful for guiding treatment
decisions, determining eligibility for trials, and making
disease-management decisions. Endopredict as a prognostic test has been
rigorously evaluated and shown to be clinically valid, even thought it
does not predict response to any specific therapy.

The use of omics technologies for drug development is beyond the scope
of this review. High dimensional omics data can be used to identify
specific molecular targets as potential mechnisms for drug development.
(3) reviews some of the regulatory issues involved in validating genomic
markers for drug targets.

Here we review the statistical issues in each of the steps for the
evaluation of omics-based tests in clinical trials. The tests can be
prognostic, predictive, or both. The long road to implementing a test in
a clinical trial starts with analytical validation, that is,
demonstrating that the omics-based assay accurately and reproducibly
measures the molecular quantities. After the assay performance is
established comes the test development and preliminary evaluation. This
involves reducing the high-dimensional data into a one-dimensional
quantity that will be used to make a decision. This one-dimensional
quantity is often a risk score: an estimate of the probability of a
speicific clinical outcome. It is necessary to establish the clinical
validity of this risk score, that is, demonstrate that the risk score is
independently associated with clinical outcome. Care must be taken to
completely separate the development of the risk score from the
evaluation, otherwise estimates can be optimistically biased. Finally,
the risk score must be translated into a binary decision, often using a
threshold. It remains to demonstrate that the use of the test to make
this decision improves patient outcomes.

\section{Analytical validation}\label{analytical-validation}

\section{Test development and preliminary
evaluation}\label{test-development-and-preliminary-evaluation}

\section{Demonstrating clinical
utility}\label{demonstrating-clinical-utility}

\section{Concluding remarks}\label{concluding-remarks}

\section*{References}\label{references}
\addcontentsline{toc}{section}{References}

1. McShane LM, Polley M-YC. Development of omics-based clinical tests
for prognosis and therapy selection: The challenge of achieving
statistical robustness and clinical utility. Clinical Trials. SAGE
Publications; 2013;10(5):653--65.

2. Filipits M, Rudas M, Jakesz R, Dubsky P, Fitzal F, Singer CF, et al.
A new molecular predictor of distant recurrence in eR-positive,
hER2-negative breast cancer adds independent information to conventional
clinical risk factors. Clinical Cancer Research. American Association
for Cancer Research; 2011;17(18):6012--20.

3. Goodsaid F, Frueh F. Process map proposal for the validation of
genomic biomarkers. Pharmacogenomics. 2006;7(5):773--82.

\bibliography{../Background/bookchapter}

\end{document}